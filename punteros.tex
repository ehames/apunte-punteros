% !TEX encoding = UTF-8 Unicode
\documentclass[12pt,a4paper,spanish]{article}
\usepackage{graphics}
\usepackage{amsmath}
\usepackage{amsfonts}
\usepackage{amssymb}
\usepackage{babel}
\usepackage{url}

\begin{document}

\title{Punteros, Arreglos y Cadenas en \textsf{C}}
\author{Edgardo Hames \\ \small{Revisado y corregido por Natalia B. Bidart, 13-08-2006}
 \\ \small{y después por Daniel F. Moisset, 24-08-2007}}
\date{}

\maketitle
\makeatother

\subsection*{Punteros y direcciones}

\textit{Organización de la memoria:} Una computadora típica tiene un arreglo de
celdas de memoria numeradas o direccionables que pueden ser manipuladas
individualmente o en grupos contiguos. Una situación común es que un byte sea un
caracter (\verb+char+), un par de celdas de un byte sea tratado como un entero
corto (\verb+short+) y cuatro bytes adyacentes formen un entero largo (\verb+long+).
Un \textit{puntero} es una variable que contiene la dirección de una variable. El
uso de punteros es de más bajo nivel, y ofrece más expresividad. A veces conduce
a un código más compacto y en algunos casos más eficiente; la desventaja es que
sin cuidado, es fácil cometer errores.

\begin{figure}[htbp]
\begin{center}
\includegraphics{memory.eps}
\caption{Organización de la memoria}
\label{fig: mem_org}
\end{center}
\end{figure}

\textit{Operadores:}
\begin{itemize}

\item El operador unario \texttt{\&} retorna la dirección de memoria de
una variable, por lo tanto:

\begin{verbatim}
p = &i;
\end{verbatim}

asigna la dirección de la variable \verb|i| a la variable puntero
\verb|p| y se dice que \verb|p| ``apunta a'' \verb|i|. Sólo se puede
aplicar a variables y elementos de arreglos\footnote{es decir, a cualquier cosa
que indique una celda de memoria. Comparese con otros operadores como que
pueden aplicarse a cualquier cosa que tenga un valor}. Ver Fig. \ref{fig:
mem_org}.

\item El operador unario \verb|*| es el operador de
\textit{indirección} o de \textit{dereferenciación}; cuando se
aplica a un puntero, accede al contenido de la memoria apuntada por
ese puntero. \verb+Ver ejemplo ej1.c+.

\begin{verbatim}
j = *p; /* j tiene ahora el valor de i */
\end{verbatim}

\item El lenguaje C nos permite comprobar los tipos de los objetos apuntados
por punteros. Esto se logra definiendo para cada tipo $T$ (por ejemplo
\verb|int|), un tipo para los puntores asociados (por ejemplo, puntero a 
\verb|int|). Para definir una variable que apunte a celdas con valores
de tipo $T$, se declara esta como si fuera de tipo $T$, pero anteponiendo
el caracter \verb|*| al nombre de la variable. Por ejemplo:

\begin{verbatim}
int *p;
\end{verbatim}

Esta notación intenta ser mnemónica, nos dice que la expresión
\verb|*p| es un \verb|int|. Aunque cada puntero apunta a un tipo
de dato específico, tambien hay un tipo llamado \verb|void *| (sería un
``puntero a void'' que puede contener cualquier tipo de punteros.

\item Como cualquier otra variable, es conveniente inicializar un
puntero en su declaración. Si el valor no se conocerá hasta más adelante
en el código, una buena convención es usar \verb+NULL+. \verb|NULL| es un valor
especial de puntero que no referencia a ninguna celda particular.

\begin{verbatim}
int *p = NULL;
\end{verbatim}

\item Como los punteros son variables, pueden ser usados sin ser
dereferenciados. Por ejemplo, si \verb+p+ y \verb+q+ son punteros a
\verb+int+,

\begin{verbatim}
p = q;
\end{verbatim}

copia el contenido de \verb+q+ en \verb+p+, o sea, copia la
dirección a la que apunta \verb+q+ en \verb+p+, haciendo que
\verb+p+ apunte a lo mismo que apunta \verb+q+.

\item Los operadores \verb+&+ y \verb+*+ asocian con mayor precedencia
que los operadores aritméticos. Por lo tanto, cualquiera de las
siguientes instrucciones incrementa en 1 el contenido de memoria
apuntada por \verb+p+.

\begin{verbatim}
*p = *p + 1;
*p += 1;
++*p;
(*p)++; /* Notar el uso de paréntesis. */
\end{verbatim}

\textbf{¡Cuidado!} Los operadores unarios \verb+*+ y \verb-++- asocian
de derecha a izquierda, por lo tanto, el uso de paréntesis suele ser
necesario.

\item El operador \verb-++- aplicado a un puntero hace que apunte a la
celda siguiente. Esto suele ser útil al manipular arrays, ya que C nos garantiza
que las celdas de un array son contiguas. \verb+Ver ejemplo ej2.c+. No es útil
en otros casos, ya que (excepto por el caso del array) una implementación de C
puede elegir asignar cual celda corresponde a cual variable en forma bastante
arbitraria.

\item En \textsf{C} el paso de parámetros a funciones es
por valor, cuando deseamos que una función pueda modificar una variable a 
elección del del llamador debemos usar punteros. Ver Fig. \ref{fig:ref_pass}.
\verb+Ver ejemplo ej3.c+

\begin{figure}[htbp]
\begin{center}
\includegraphics{swap.eps}
\caption{Paso de referencias a funciones.}
\label{fig:ref_pass}
\end{center}
\end{figure}

\end{itemize}

\subsection*{Punteros y Arreglos}

En \textsf{C}, hay una relación muy fuerte entre punteros y arreglos.
Cualquier operación que pueda ser lograda indexando un arreglo también
pude ser conseguida con punteros. La declaración:

\begin{verbatim}
int a[10];
\end{verbatim}

define un arreglo de tamaño 10, o sea un bloque de 10 enteros
consecutivos que se acceden a través de \verb+a[0]+, \verb+a[1]+,
\ldots, \verb+a[9]+. Ver Fig. \ref{fig:array}.

\begin{figure}[htbp]
\begin{center}
\includegraphics{array.eps}
\caption{Arreglo de 10 caracteres.}
\label{fig:array}
\end{center}
\end{figure}

La notación \verb+a[i]+ hace referencia al \verb+i+-ésimo elemento del arreglo.
Supongamos que \verb+pa+ es un puntero a enteros, declarado como

\begin{verbatim}
int *pa;
\end{verbatim}

entonces la asignación

\begin{verbatim}
pa = &a[0];
\end{verbatim}

hace que \verb+pa+ apunte al elemento cero de \verb+a+, o sea \verb+pa+
contiene la dirección de \verb+a[0]+. El nombre de un arreglo es un
puntero a su primer elemento (con lo cual podemos escribir el ejemplo anterior
como \verb|pa = a|).

Si \verb+pa+ apunta a un elemento particular de un arreglo, entonces por
definición \verb-pa+1- apunta al siguiente elemento, \verb-pa+i- apunta
\verb+i+ elementos más adelante y \verb+pa-1+ apunta \verb+i+ elementos
antes. Por lo tanto si \verb+pa+ apunta a \verb+a[0]+,

\begin{verbatim}
*(pa + 1);
\end{verbatim}

referencia al contenido de \verb+a[1]+, \verb-pa+i- es la dirección de
\verb+a[i]+ y \verb-*(pa+i)- es el contenido de \verb+a[i]+.

Hagamos algunas cuentas sencillas \ldots

\[ a[i] = *(pa + i) = *(i + pa) = i[a] \]

¿Será cierto lo que nos dicen nuestras clases de álgebra?
\verb+Ver ejemplo ej4.c+.

\subsection*{Cadenas de caracteres}

\begin{itemize}

\item Una variable de tipo \texttt{char} sólo puede almacenar un único
caracter.

\begin{verbatim}
char c = 'A';
\end{verbatim}

\item Un \textit{string} es una secuencia de caracteres. Por ej: "Hallo,
Welt!"

\item Una mala noticia: \textbf{\textsf{C} no soporta el tipo string}.

\item En \textsf{C} las operaciones sobre \textit{strings} en realidad trabajan
sobre arreglos de caracteres que siguen algunas convenciones. Esencialmente
van los caracteres en el orden en el que se leen, el string no puede contener
el caracter especial \verb|'\0'|, y al final de la cadena se agrega un
\verb|'\0'| (que no es parte de la cadena, pero que muchas funciones utilizan
para saber donde termina el string).

\item Generalmente, una variable de tipo \textit{string} se declara como
un puntero a \verb+char+ (en realidad, un puntero al primer elemento de un
array de \verb|char|, que ya vimos que en C es casi lo mismo). Ejemplos:

\begin{verbatim}
char saludo_arr[] = "Salut, mundi.";
char *saludo_ptr = "Salut, mundi.";
\end{verbatim}

Hay una diferencia importante entre las dos declaraciones mostradas. En
la primera, \verb+saludo_arr+ es un array, de tamaño 13+1 (el 1 suma por
el caracter terminador \verb|'\0'|). Los caracteres del array pueden ser
cambiados, pero \verb+saludo_arr+ siempre va a apuntar a la misma zona
de memoria.

En cambio, el puntero \verb+saludo_ptr+ puede ser eventualmente
re-apuntado a otra zona de memoria, pero si se trata de modificar el
contenido del string se obtiene un resultado indefinido\footnote{Cuando el
estándar de C dice ``esto produce un resultado indefinido'' quiere decir que
su programa puede fallar intermitentemente, de formas distintas, y en
condiciones poco esperadas.}.

\item Para acceder al string usamos la variable
\verb|saludo_{arr,ptr}|. Ejemplo usando \verb+printf+:

\begin{verbatim}
printf ("El contenido de saludo_ptr es: %s", saludo_ptr);
\end{verbatim}

\item Por convención, el caracter nulo \verb+\0+ marca el fin del
string.

\begin{verbatim}
char *mensaje = {`H',`o',`l',`a',`,',` ',`m',`u',`n',`d',`o',`\0'};
char *mensaje = "Hola, mundo";
\end{verbatim}

\item Aunque a veces al caracter \verb+`\0'+ se le dice ``caracter nulo'' o
inlcuso NUL (con una sola `L'!), no tiene nada que ver con
\verb+NULL+. Son valores de distintos tipos. Además \verb+`\0'+ es distinto
de \verb+`0'+ (el caracter del simbolo ``cero'' usado 2 veces en la cadena 
"007").

\begin{verbatim}
char *ptr;
char c;

if (c == '\0') {
    /* c es el caracter nulo. */
}
if (!c) {
    /* c es el caracter nulo. */
}
if (*ptr == '\0') {
    /* p apunta al caracter nulo. */
}
if (!*ptr) {
    /* p apunta a un caracter nulo. */
}

\end{verbatim}

parecido pero distinto de \ldots

\begin{verbatim}
char *ptr;

if (!ptr) {
    /* p es un puntero nulo. */
}
\end{verbatim}

\end{itemize}

\subsection*{Problemas con la librería de cadenas de \textsf{C}}

\begin{itemize}

\item El uso de \verb+\0+ para denotar el fin del string implica que
determinar la longitud de la cadena es una operación de orden lineal
$O(n)$ cuando puede ser constante $O(1)$.

\item Se impone una interpretación para valor del caracter \verb+\0+.
Por lo tanto, \verb+\0+ no puede formar parte de un string.

\item \verb+fgets+ tiene la inusual semántica de ignorar los
\verb+\0+ que encuentra antes del caracter \verb+\n+.

\item No hay administración de memoria y las operaciones provistas
(\verb+strcpy+, \verb+strcat+, \verb+sprintf+, etc.) son lugares comunes
para el desbordamiento de buffers\footnote{Un desbordamiento de buffer es cuando
un programa escribe fuera de los limites de un array. Muchas veces pueden
explotarse para tomar el control del programa y hacer que haga cosas que el
programador no quería, como mandar SPAM desde nuestra cuenta de correo}.

\item Pasar \verb+NULL+ a la librería de strings de \textsf{C} provoca
un acceso a puntero nulo.

\end{itemize}

Hay un canción que hace \textit{referencia} a los strings en \textsf{C},
se llama \textit{``Lo que ves es lo que hay''}\footnote{Charly García,
Álbum ``El aguante'' :-)}.

\subsection*{La destreza y la memoria son buenas si van en yunta}

\begin{itemize}

\item Arrays de caracteres declarados como \verb+const+ no pueden ser
modificados. Por lo tanto, es de muy buen estilo declarar strings que no
deben ser modificados con tipo \verb+const char *+.

\item La memoria asignada para un array de caracteres puede extenderse
más allá del caracter nulo. A veces eso es útil (por ejemplo si sabemos que vamos
a querer agregar caracteres)

\begin{verbatim}
char ptr[20];

strcpy(ptr, "Hola, Mundo");
/* Sobran 2^3 caracteres: ptr[12] - ptr[19]. */
\end{verbatim}

\item Una fuente muy común de \textit{bugs} es intentar poner más
caracteres de los que caben en el espacio asignado. Recordemos hacer
lugar para \verb+\0+! Estos \textit{bugs} son una de las causas que
pueden generar que la ejecución de un programa termine anormalmente con un
\verb+segmentation fault+.

\item Las funciones de la biblioteca NO toman en cuenta el tamaño de la
memoria asignada. \textsf{C} asume que el programador sabe lo que
hace... :-) \verb+Ver ejemplo ej5.c+

\item Para muchas funciones de la forma \verb+strXXX+ existe una versión
\verb+strnXXX+ que opera sobre los \verb+n+ primeros caracteres del array.
Es recomendable usar estas funciones para evitar problemas desbordando arreglos.

\item Notar que podemos usar un índice mayor a la longitud del string
sin que nos dé error. \verb+Ver ejemplo ej5.c+.

\begin{verbatim}
/* Probar en Haskell: "Hello, World!" !! 20 */
\end{verbatim}

\end{itemize}

\subsection*{Funciones útiles}

Estas funciones están documentadas en la sección 3 de las \textit{man
pages}.

\begin{itemize}

\item Funciones que operan sobre caracteres: \footnote{Si operan sobre
caracteres, ¿por qué no hay \texttt{char} en los prototipos?}
\footnote{¿no deberían retornar \texttt{bool} o algo similar?}

\begin{verbatim}
#include <ctype.h>

int isalnum (int c);
int isalpha (int c);
int isascii (int c);
int isblank (int c);
int iscntrl (int c);
int isdigit (int c);
int isgraph (int c);
int islower (int c);
int isprint (int c);
int ispunct (int c);
int isspace (int c);
int isupper (int c);
int isxdigit (int c);
\end{verbatim}

\item Funciones que operan sobre strings:

\begin{verbatim}
#include <string.h>

char *strcpy(char *dest, const char *orig);
char *strcat(char *dest, const char *src);
char *strstr(const char *haystack, const char *needle);
size_t strspn(const char *s, const char *acepta);
size_t strcspn(const char *s, const char *rechaza);
\end{verbatim}

\end{itemize}

\section*{Referencias}

\begin{itemize}
\item The \textsf{C} Programming Language - Brian Kernighan, Dennis Ritchie
\item \url{http://www.harpercollege.edu/bus-ss/cis/166/mmckenzi/contents.htm}
\item \url{http://www.harpercollege.edu/bus-ss/cis/166/mmckenzi/lect12/l12.htm}
\item \url{http://bstring.sourceforge.net}
\item \url{http://www.cs.princeton.edu/courses/archive/spring02/cs217/asgts/ish/ish.html}
\item \url{http://www.cs.princeton.edu/courses/archive/spring02/cs217/asgts/ish/ishhints.html}
\end{itemize}

\end{document}
